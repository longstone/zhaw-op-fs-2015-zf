\section{W13 - Quality Management}
\subsubsection{Repetition: The 7+1 Types of Waste (= muda)}
\subsubsection{Revision: How to Become Lean – Example of Toolset }
\ref{sssec:becomelean}
\subsection{Quality and Errors}
\subsubsection{What is Quality in Production? }
\subsubsection{What is Quality in Service?}
\subsubsection{Revision: Quality means different things in different operations}
\subsubsection{Perceived Quality is Governed by the Gap Between Customers’ Expectations and Their Perceptions of the Product or Service}
\subsubsection{High Quality Puts Costs down and Revenue up}
\subsubsection{A ‘Gap’ Model of Quality}
\subsubsection{The Perception – Expectation Gap}
\subsubsection{Conformance to Specification}
\subsubsection{Quality Characteristics of Goods and Services}
\subsubsection{Quality Problem – Whose Fault Was It? }
\subsubsection{Why Examine and Change Processes? Among Other Things, to Increase Quality and Reduce Errors}
''The majority of (medical) errors do not result from
individual recklessness or the actions of a particular
group - this is not a ''bad apple`` problem.
More commonly, errors are caused by faulty systems,
processes, and conditions that lead people to make
mistakes or fail to prevent them''
\subsubsection{What is the Magnitude of Errors? }
Typical range of cost of poor quality
(COPQ as \% of Sales):\\
\begin{center}
\begin{tabular}{|l|r|}
	\hline Manufacturing & 20-30\% \\ 
	\hline Services & 30-40\% \\ 
	\hline Software & 40-65\% \\ 
	\hline 
\end{tabular} 
\end{center}
\subsubsection{TQM Can be Viewed as a Natural Extension of Earlier Approaches to Quality Management}
\subsubsection{Total Quality Management: What Does TQM Include?}
1. Includes all parts of the organization
2. Includes all staff of the organization
3. Includes consideration of all costs
4. Includes every opportunity to get things right
5. Includes all the systems that affect quality
6. And it never stops!
\subsubsection{The Internal Customer–Supplier Concept Involves Understanding the Relationship Between Processes}
\subsubsection{The Traditional Cost-of-Quality Model}
\subsubsection{The Traditional Cost-of-Quality Model with Adjustments to Reflect Criticism of TQM}
\subsubsection{How to Prevent Errors Cheaply: Poka-Yoke}
-> Idiotensicher
\subsubsection{Poka-Yoke or Bullet-Proof Assembly for Kite Surfers}
\subsubsection{Poka-Yoke is a Japanese Term Meaning ''Fail-Safing'' or ''Mistake-Proofing'' }
\subsubsection{Increasing the Effort Spent on Preventing Errors from Occurring in the First Place Leads to a more than Equivalent Reduction in other	Cost Categories}
\subsubsection{The Cost of Rectifying Errors increases the Longer the Errors Remain Uncorrected in the Development and Launch Process}
\subsubsection{TQM in Services}
\subsubsection{How to Assure Quality in Services? TQM at UPS}
\subsubsection{How to Measure and Prove Quality? Examples of Certification and International Codes }
\subsubsection{EFQM ‘Business excellence’ Model}
\subsubsection{EFQM Characteristics }
Self assessment (it’s possible to adapt weighting system
of the 9 categories)
Excellent results can only be achieved if the interests of all
involved parties are taken into account
Methods: benchmarking, Kaizen, PDCA Cycle, etc.
''Committed to Excellence'': 3 successful implemented
improvement concepts and audit through validator (2-year
shelf life)
''Recognized to Excellence'': Extensive self assessment and
audit; ''Stars Award'' depending on points achieved
Competition in Switzerland: ESPRIX – Swiss Excellence
Award
\subsection{Quality Assurance}
\subsubsection{Quality Management in Daily Life: Wasabi-nuts}
\subsubsection{Costs of Poor Performance / Quality}
\begin{itemize}
	\item \textbf{Defects}\index{Defects} any instance when a proces fails to satisfy ist customer
	\item \textbf{Prevention costs}\index{Prevention costs} are associate with preventing defects before they happen
	\item \textbf{Appraisal costs}\index{Appraisal costs} are incurred when the firm assesses the performance level of the processes
	\item \textbf{Internal failure costs}\index{Internal failure costs}, result from defects that are discovered during production of a service or products
	\item \textbf{External failure costs}\index{External failure costs}, arise when a defect is discovered after the customer receives the service or product
\item
\end{itemize}
\subsubsection{Important Analytical QA Methods: QA Instruments in Practice }
\subsubsection{SPC - Natural vs. assignable causes of variation}
\subsubsection{Natural Variations vs. Assignable Variations}
\subsubsection{Histogram: Standard Deviation = Measure for Deviation}
\subsection{Six Sigma}
\subsubsection{Six Sigma: Basic Information}
\subsubsection{Process Variation and Its Effect on Process Defects per Million Opportunities (DPMO)}
\subsubsection{Low Process Variation Allows Changes in Process Performance to be Readily Detected}
\subsubsection{Six Sigma is Based on 3 Main Elements}