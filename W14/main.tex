\section{Berechnung OEE / OPE (Alle Wege)}
\subsection{OEE: Unterscheidung Sequentiell und Parallel}
\section{OPE}
Zusammenhang OPE Zykluszeit \& Balancing Loss (Balancing Loss ist auf Basis ist Zykluszeit und nicht Kapazit\"aten)
\ref{balancingLoss} Berechnung Balancing loss
\section{Heijunka -> Unterscheidung Heijunka viele Maschinen und nur eine Maschine}
http://de.wikipedia.org/wiki/Heijunka
Bei Heijunka geht es um die weitgehende Harmonisierung des Produktionsflusses durch einen mengenm\"assigen Ausgleich. Es handelt sich um eine Weiterf\"uhrung des \index{Heikinka}Heikinka, der nivellierten Produktion, bei welcher der bereits feste Produktionszyklus \"ofter als einmal am Tag wiederholt wird. Ohne Nivellieren kann kein Synchrones Produktionssystem geschaffen werden.

Warteschlangen und damit \textbf{Liege- und Transportzeite}n sollen beim Heijunka weitgehend \textbf{vermieden} werden. Eine Fliessproduktion (Continuous-Flow-Manufacturing) mit kurzen Transportwegen ist dazu Voraussetzung. Das Konzept ist vor allem angesichts komplexer, mehrstufiger Produktion von hoher Bedeutung. Die jeweiligen Engp\"asse wirken hier limitierend f\"ur das ganze System (Ausgleichsgesetz der Planung) und erzeugen damit zugleich bei allen anderen Teilen Verschwendung (siehe auch: Muda).
\section{Supply Chain Management: Wahl der Transportwege (auch Quantitativ angewendet -> siehe \"Ubung Smartphone)}
\section{Qualitativ verstehen: Produkt \& Prozessentwicklung}
\section{EOQ \& EBQ: Gleichungen aufl\"osen}
\ref{EOQ} EOQ
Economic order quantity (EOQ) is the level of inventory that minimizes the total inventory holding costs and ordering costs.\\
	Economic batch quantity (EBQ), also called "optimal batch quantity" or economic production quantity, is a measure used to determine the quantity of units that can be produced at minimum average costs in a given batch or production run.
\section{Statistische Prozesskontrolle: Mit Messgr\"ossen (ETC) vertraut machen die f\"ur die ermittlung der Kennzahlen wichtig sind}

\section{Layoutformen wann ist welche geeignet}
\ref{allLayouts} All layouts in one example
\section{Kapazit\"atsstrategie}
wie wird diese geplant. Wann und zu welchem Zeitpunkt wird Kapazit\"at aufgebaut (in Erwartung damgekauft wird, oder erst bei regelm\"assigen R\"uckst\"anden)
\section{5 Leistungsziele -> Bedeutung und abh\"angigkeit}
\ref{FivePerformanceGoals} 5 Performance Goals
\section{Operationsstrategie -> Bedeutung und wozu dient es}
\ref{whatIsStrategy}