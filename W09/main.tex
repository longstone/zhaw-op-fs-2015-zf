\section{W09 - Capacity Management}
\subsection{Introduction}
\subsubsection{What Capacity Does this Operation Have?}
\subsubsection{What is Capacity? }
\begin{itemize}
	\item Capacity is in the static, physical sense the fixed volume of a container,
	or the space in a building, or the seats in a plane, theater, etc.
	\item The definition of the capacity of an operation is the maximum level of
	value-added activity over a period of time that the process can achieve
	under normal operating conditions
\end{itemize}
\subsubsection{Capacity Constraints}
\begin{itemize}
	\item The parts of the operation that are operating at their capacity ‘ceiling’
	are capacity constraints for the whole operation.
	\item These constraints are often referred to as ‘bottlenecks’. Depending on
	the nature of demand, different parts of an operation may be pushed to
	their capacity ceiling and act as a bottleneck.
\end{itemize}
\subsubsection{Capacity Metaphor }
\subsubsection{Medium- and Short-Term Capacity Planning}
Medium-term capacity planning involves an
assessment of demand forecasts over a period of 2 to
18 months, during which time planned output can be
varied, for example, by changing the number of hours
the equipment is used \\
Short-term capacity planning adjusts the resources
with a time horizon of days or hours, e.g. the services
in a garden restaurant
\subsubsection{Basic Questions in Capacity Planning}
\begin{itemize}
	\item Demand quantity
	\item Dependencies of the demand
	\item Demand seasonality 
	\item Limits of supply 
	\item Storing capabilities of offered product 
	\item Available capabilities
\end{itemize}
\subsubsection{Good forecasts are essential for effective capacity planning}
\subsubsection{Causes of Seasonality}
\subsubsection{Demand Fluctuation Hotel}
\subsubsection{Demand Fluctuation Retailer}
\subsubsection{Demand Fluctuation Aluminium manufacturer}
\subsection{Planning Capacity}
\subsubsection{Ways of Reconciling Capacity and Demand}
\subsubsection{Level Capacity Plan with Inventory Built up}
\subsubsection{Chase Demand Plan}
\subsubsection{Adjust Personnel Resources to Match Demand}
\subsubsection{Combination of Different Capacity Planning Strategies}
\subsubsection{Create a Capacity Plan: Cumulative Demand}
\subsubsection{Create a Capacity Plan: Cumulative Demand}
\subsubsection{Cumulative Representations: Level Capacity Plan}
\subsubsection{Cumulative Representations: Level Capacity Plan, Starting without an Inventory}
\subsubsection{Cumulative Representations: Level Capacity Plan, Starting with an Inventory}
\subsubsection{Cumulative Representations: Chase Demand Plan, Starting with an Inventory}
\subsubsection{Exercise: Capacity Planning}
\subsubsection{Challenge: The Balance of Capacity}
\subsubsection{Capacity-Leading or -Lagging Strategies}
\subsubsection{Small or Big Lots?}
\subsection{Capacity and Queues}
\subsubsection{Simple queuing system}
\subsubsection{Simple Queuing System (Continued)}
\subsubsection{Capacity vs. Lead-Time}\subsubsection{Goal: Reduce the Variance of the System }
