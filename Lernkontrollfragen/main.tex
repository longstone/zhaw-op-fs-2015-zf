\section{Lernkontrollfragen}
\subsection{4V\index{4V}}
Where is a low visibility (with regard to the 4V concept) typically given and advantageous?
\begin{itemize}
	\item \textbf{for selling standardized and massmanufactured products and services through e-commerce channels}
	\item in customer-specific operations
	\item for hiding confidential production know-how in area of intrusion and fire protection systems 
	\item where a fully customized production results in a time lag between placing and delivering an order
\end{itemize}
\subsection{Operations Management}
\index{Operations Management Levels}
Operations Management distinguishes between three levels:
\begin{itemize}
\item The supply network level which is concerned with the flow of goods and information between operations
\item The operations level which represents activities in a department or in a subsidiary
\item The process level which represents the flow between resources such as equipment or personnel
\end{itemize}
Which of the following statements does NOT reflect this classification?
\begin{itemize}
	\item \textbf{A process optimization initiative of workflows of a production line refers to the operations level}
	\item a mistake in the clearing and settlement process of a credit card payment affects the entire supply network
	\item An operations strategy considers all three levels
	\item Resource planning is executed at the ''process level''
	\item Service operations share information or information products along the supply chain
\end{itemize}
\subsection{Produktivit\"at\index{Produktivit\"at}}
According to a US study discussed in the lecture, which factor has the highest impact on productivity?\\
\begin{itemize}
\item Suppliers
\item Employees
\item \textbf{Management} 
\item Kunden
\item Kapital
\end{itemize}
\subsection{Qualifizierende Faktoren}\index{Qualifizierende Faktoren}
Qualifying factors incorporate the following characteristic\\
\begin{itemize}
\item they improve proportionally as the performance increases
\item are not relevant for the performance
\item determine measuring values on a performance scale
\item is a measure for an increasing performance advantage
\item \textbf{the qualifying performance level must be achieved. Overperformance will not be rewarded}
\end{itemize}
\subsection{Operations Entwicklungsstufen\index{Operations!Entwicklungsstufen}}
How may neutral development stages in area of operations be differentiated from supportive development stages?
\begin{itemize}
\item neutral development stages ensure a competitive advantage
\item by adopting "best practice", operations become supportive
\item  \textbf{supportive development stages require a leading position in area of operations Richtig}
\item  a neutral development stage ensures a competitive advantage
\item  There is no differentiation
\end{itemize}
\subsection{Operations Strategie\index{Operations!Strategie}}
Which of the following performance objectives is \textbf{not} considered part of an operations strategy?
\begin{itemize}
\item   The reliability of a service operation.
\item   The delivery time for a highly individualized sports car.
\item   The service quality of an aviation school in Lugano.
\item   The cost efficiency of an operation compared to the competition.
\item  \textbf{ The effectiveness of an internet presence for a customer segment.}
\end{itemize}
\subsection{Time to Market\index{Time to Market}}
The chances of successfully launching new products or services increases with faster time-to-market. Which factor might prevent such faster time-to-market?
\begin{itemize}
\item   \textbf{The management team and sales advisors introduce innovative design ideas shortly before the launch of the product or the service.}
\item   Depending on the particular product or service development process, the team is organized and a project manager with the overall responsibility is assigned.
\item   Uncertainties and different concepts of products or services are clarified during the early conceptual phase of the product or service development process.
\item   Various phases of the product or service development process occur simultaneously. Falsch
\item Supporting tools, such as CAD or prototypes, are used during the product or service development process.
\end{itemize}
\subsection{Push \& Pull\index{Push \& Pull}}
In the material replenishment process, we differentiate between push and pull systems. Regarding these push and pull systems, which of the following statements is correct?
\begin{itemize}
\item If the supplier reorders in a planned and periodical manner, the process always corresponds to a pull system. 
\item A change towards a pull system causes out-of-stock situation at customer's plant. 
\item An empty inventory at the supplier's plant only occurs in the push system. 
\item Very high inventory levels at the customer's plant typically occurs in a pull system. 
\item\textbf{ If the customer initiates the reorder of materials, the process corresponds to a pull system.}
\end{itemize}
\subsection{Push \& Pull}
In the containership game several groups have changed from a push system to a pull system. What is generally \textbf{no} effect of a transition from a push system to a pull system?

\begin{itemize}
	\item Waste (Muda) decreases. Falsch
	\item The production of ships is completed just-in-time.
	\item The customer dictates the cycle time.
	\item\textbf{ The stock level increases.}
	\item The volatile customer demand is better met.
\end{itemize}
\subsection{Little's Law\index{Little's Law}}
You are responsible for the business development of the Starbucks chain in Switzerland. During the opening day of a new coffee shop, you observe that 200 customers enter the shop during the first two hours. During the third and fourth hour another 200 customers enter the shop. A further analysis reveals that 45 customers are present on average during the third hour. What is the average length of a stay of a customer? Assume that the output and input in such a black box need to be balanced. 
\begin{itemize}
\item The average length of a stay cannot be calculated with the data provided above.  
\item 27 min
\item  13.5 min
\item  0.27 h
\item  45 min
\end{itemize}
\subsection{Cycle Time\index{Cycle Time}}
A cinema has 200 visitors. During a 20 minutes break, 50 visitors remain seated, 50 visitors leave the cinema because of the bad movie and 100 visitors purchase a snack. The cinema provides 10 snack sales counters. On average a snack sale lasts 90 seconds. What is the cycle time if either 5 or 10 snack sales counters are open?

\begin{itemize}
\item in both cases 0.2 min.
\item \textbf{ 5 counters: 0.3 min. 10 counters: 0.15 min.}
\item  5 counters: 0.2 min. 1 O counters: 0.1 min.
\item  in both cases 0.15 min.
\item  5 counters: 0.3 min. 10 counters: 0.6 min.
\end{itemize}

\subsection{Upstream Integration}
\index{Upstream!Integration}
\index{Integration!Upstream}
Why do companies choose to integrate upstreams?\\
\begin{itemize}
\item Because organizations can therefore offer a broader product offering.
\item Because organizations can therefore easier integrate downstreams.
\item Because organizations can better control marketing activities.
\item \textbf{Because organizations strive for controlling strategic important resources (e.g. commodities).}
\item ecause organizations become independent of raw material prices.
\end{itemize}
\subsection{Bullwhip\index{Bullwhip}}
Felix, the CEO of Chocolate Inc. makes up thoughts on the Bullwhip effect regarding his chocolate bars "Vanilla-Cocoa-Dream". He instructed his team to develop ideas to reduce the probability of  the emergence of the Bullwhip effect in the supply chain. The following ideas and justifications were presented to him. 
What combination of an idea and its justification \textbf{is not correct and has no influence for preventing} the Bullwhip effect?
\begin{itemize}
	\item \textbf{. / Introducing the promotion ''+ 30\% content for the same price'' - to reduce the variance of the final demand. }
	\item Direct import of the cocoa beans from the farmers in Costa Rica - the exclusion of any intermediary is necessary because these firms have different ERP systems which are not campatible to exchange information. 
	\item Installation of an automatic ordering system - IT systems react less irrational to low stocks and therefore prevent unusually high orders. 
	\item Daily orders of sugar from the suppliers - regular orders result in lower variance regarding the order quantity rather than larger batches.
	\item Implementing an agile and fast supply chain for the vanilla beans - slow supply chains reinforce the bullwhip effect.
\end{itemize}
\subsection{Bullwhip\index{Bullwhip}}
Which of the five statements is wrong with regard to the bullwhip effect?
\begin{itemize}
	\item The bullwhip effect can be reduced by keeping (re-) selling prices at a constant level. Falsch
	\item One reason for the occurrence of the bullwhip stems from overreactions related to order backlogs.
	\item \textbf{The bullwhip effect can be observed while demand is rising. The Bullwhip effect in practice cannot be observed while demand is falling.}
	\item A further reason for the occurrence of the bullwhip effect is the lack of communication within the supply chain.
	\item The occurrence of the bullwhip effect is not limited to complete supply chains. The bullwhip effect can also be observed inside individual companies.
\end{itemize}

\subsection{Bullwhip\index{Bullwhip}}
Every summer, the cheese industry faces the same challenge in producing fresh cheese: Customers ask for an above-average amount of fresh cheese. This seasonal fluctuation causes shortages of milk, and consequently leads to bullwhip effects in the supply chain. Which of the following strategies would you not recommend during summer?

\begin{itemize}
\item During summer months, cheese producers should change their logistics systems from mono-article palettes to palettes with mixed articles. This reduces the delivery frequency for each article.
\item \textbf{Retailers should establish shortages of fresh cheese by only putting fewer articles in the shelves. This signals customers that the products are scarce.}
\item etailers could establish vendor-managed inventory contracts with the cheese producers. By doing so, the producers would know the inventory level and could forecast more precisely which articles need to be replenished. 
\item During summer months, it is not advisable to promote fresh cheese. Promotions would increase the demand for fresh cheese additionally.
\item Retailers should establish an electronic data interchange with the cheese producers. Consequently, the companies in the supply chain know which products have the highest demand. Moreover, this enables prioritizing the production according to the current sales.
\end{itemize}
\subsection{Bullwhip\index{Bullwhip}}
You were charged to reduce the bullwhip-effect in the supply chain. Which of the
following statements regarding the bullwhip-effect is \textbf{wrong}?
\begin{itemize}
	\item \textbf{Sales of our core products (such as Cheeseburgers, Coke, Diet Coke) are not affected by the bullwhip effect since these products have a very long life cycles. }
	\item Infrequent orders of large batches can result in a bullwhip effect.
	\item One of the main reasons for the bullwhip effect is that we have provided only partial and incomplete sales figures to our suppliers.
	\item By means of a vendor managed inventory the bullwhip effect can be reduced.
	\item The application of supply chain management software reduces the bullwhip effect.
\end{itemize}
\subsection{Supply Network\index{Supply Network} }
Which of the following statements regarding supply network design is correct?
\begin{itemize}
	\item The price fixing at the commodity exchange significantly affects international supply chains because high commodity prices lead to increased transportation costs. 
	\item Second-tier suppliers are external suppliers and require a cooperation with a logistics service provider. 
	\item  Risk management in supply chain design is confined to first-tier customers and suppliers. 
	\item \textbf{Organizations should outsource activities which are of no strategic value and which are produced internally at significantly higher costs than market prices.  }
	\item  A vertical integration upstream in the supply chain provides close access to customers and will lead to a better controlled supply chain accordingly. 
\end{itemize}

\subsection{Supply Chain Management\index{Supply Chain Management}}
Which of the following statements does \textbf{not} apply in Supply Chain Management?
\begin{itemize}
\item Suppy Chain Management is concerned about the material and informations flow
\item \textbf{An organization that covers various steps in the supply chain from upstream to downstream is horizontally integrated}
\item Vertical integration means the additional integration of upstream and/or downstream value chains 
\item Horizontal integration means the integration of additional parallel value creation steps
\item Downstream (related to vertical Integration) symbolizes the flow of materials between the manufacturer and the customer
\end{itemize}
\subsection{Supply Chain Management\index{Supply Chain Management}}
The Fast-Food Restaurant Chain ''Tim\&Friends'' is listed on the NASDAQ Stockexchange. "Tim\&Friends" has more than 500 outlets and is present in many global markets. Last year "Tim\&Friends" could not sell their Premium-Burger ,,King Triple Cheese" due to a severe capacity constraint in the purchasing department. As a result you were mandated by
the "Tim\&Friends" management team to establish an internal Supply Chain Management (SCM) department.
One of the directors of the "Tim\&Friends" is not convinced that SCM is important. Based on your knowledge regarding the study of Kevin B. Hendricks (see the figure below) you undertake to convince the questionable director about the importance of SCM. Which of the following answers is not correct according to Hendricks's study?
\begin{itemize}
\item  Announcements regarding supply chain problems typically have a strong impact on the share price of a listed company.
item \textbf{The share price typically declines before a supply chain problem is officially announced. One explanation is that negative effects of the glitch are previously published on the social media sites of the company.}
\item The day the announcement of a supply chain problem is made, the share price falls dramatically.
\item The share price falls rapidly. The stock market acts rationaly because such effects might have severe financial impacts on the organization. A rapid recovery of the share price is not to be expected.
\item SCM, amongst other benefits, undertakes to reduce the probability of such supply chain
problems.
\end{itemize}
\subsection{Supply Chain Management\index{Supply Chain Management}}
A fashion retailer decides to integrate its suppliers by means of EDI. What influences the payback period of the EDI implementation project most significantly?
\begin{itemize}
	\item The introduction of the most current technology (Best-of-Breed).
	\item The conclusion or signing of a service contract with the IT service company.
	\item An e-commerce solution, which facilitates communication with customers.
\item \textbf{The adoption rate of the EDI-solution by the suppliers.} 
\item The elimination of paper purchasing costs.
\end{itemize}
\subsection{Supply Chain Management\index{Supply Chain Management}}
What is the main reason for enterprises to continuously spend time and money to further develop their suppliers (Lieferantenentwicklung) ? 
\begin{itemize}
	\item Enterprises test their own Project Management skills.
	\item Enterprises seek access to industrial and trade secrets of their suppliers in order to resell those to other suppliers. 
	\item Enterprises seek to develop their own consulting competence in order to develop new business activities. 
	\item \textbf{Despite initital investments, enterprises expect a positive financial impact in the mid to long term. }
\item Enterprises can quickly reduce their cost. 
\end{itemize}
\subsection{Supply Chain Management\index{Supply Chain Management}}
You are the operations and supply chain manager of a big pharmaceutical company. Your boss tells you, that he wants to eliminate all the inventories in the entire company. Which would be the best answer to your boss?
\begin{itemize}
	\item Correct. Inventories cost money and have to be avoided under any circumstances.
	\item Not correct. There are hardly and companies which are able to produce everything just-in-time. 
	\item Correct. Other companies have optimized their inventories and have outsourced them to other companies.
	\item Not correct. Inventories cause costs, which cannot be optimized.
	\item \textbf{Not correct. Having inventories cannot be avoided in certain production processes. }
\end{itemize}
\subsection{Supply Chain Management\index{Supply Chain Management} / ERP\index{ERP}}
A toy producer wants to introduce an ERP-system. Which of the following five statements is
\textbf{wrong}?
\begin{itemize}
	\item Especially for larger companies an ERP-system facilitates the planning and controlling of
	processes throughout the company.
	\item The purpose of an ERP-system is to integrate all departments and their functions in one IT-system. A further aim is to prevent redundant storage of data. 
	\item An ERP-system is an enterprise-wide business solution which consists of different software modules (for example production and inventory control, purchasing, sales and distribution, human resources etc.).
	\item An important success factor in the development and market penetration of ERP-systems has been the rapid development of powerful computer and network systems.
	\item \textbf{ERP systems are responsible for integrating different players in the supply chain. }
\end{itemize}

\subsection{OEE\index{OEE}}
A pharmaceutical company evaluates two different layouts for the production of the frequently sold drug "Oblivion". Both layout types have the same capacity and require the same initial investment.
Layout A enables the simultaneous production of 'Oblivion' with four independent machines running in parallel mode. Layout B also uses four machines. These produce 'Oblivion' in sequential mode one after the other. It is assumed that the machines used in the layout A and in the layout B have a typical \index{OEE} of between 70\% and 90\%.
What is the respective \index{OEE} of layout A and of layout B if in one given month if all the machines have an average \index{OEE} of 70\%?
\begin{itemize}
	\item \textbf{Layout A 70\% and Layout B 24\%}\\ Parallel: $\frac{70\%+70\%+70\%+70\%}{4} = 70\%$\\ Sequentiell: $70\%\cdot70\%\cdot70\%\cdot70\% = 24\%$
	\item Both Layouts have the same OEE\index{OEE} of 70\%
	\item Both Layouts have the same \index{OEE} of 210\%
	\item Layout A 23.3\% and Layout B 34.4\%
\end{itemize}
\subsection{OEE\index{OEE}}
You work in retail banking and make a time study. The bank has an 8-hour work day. You are observing a team member in the retirement plans section. On average, he takes 51.2 minutes to service a customer. You rate this at 80 out of 100. 

Which of the following statements is true?
\begin{itemize}
	\item Basic time is 8 hours.
	\item The observed team member is 33\% slower than the average
	\item \textbf{The observed team member is 25\% slower than the average.}
	\item The basic time equals 51.2 minutes. 
	\item The basic time equals 68 minutes. 
\end{itemize}
$8h \cdot 80\% = 6.4h$ =>
$51.2min \cdot \frac{8h}{ 6.4h} = 64 min$ =>
$\frac{64min}{51.2min} = 1,25 => 25\% langsamer$

\subsection{OEE\index{OEE}}
You are starting your new job as assistant manager to the chief operating officer of a manufacturing company. Production works in shifts of 8-hours each. All employees can take two 15-minute breaks and one 30-minute meal break per shift. During the breaks the production is stopped. The average downtime of the welding machine is 47 minutes per shift. Its specifications say that it should ideally be run at 60 parts per minute. You analyze that on average 19,271 pieces are produced per shift, out of which 423 do not pass quality control.

Calculate the OEE. 
\begin{itemize}
	\item The Overall Equipment Effectiveness is between 60 and 65\%. 
	\item The Overall Equipment Effectiveness is between 50 and 55\%. 
	\item None of the answers is correct. 
	\item \textbf{The Overall Equipment Effectiveness is between 70 and 75\%}\\
	$8h = 480 min$
	Abzug: 60 min Pause, 47 min Schweissen, 7 min Qualit\"atsverlust $\frac{423Teile}{60 Teile pro Minute}$
	$OEE = \frac{377 min}{480 min} = 76\%$
	\item Based on the information mentioned in the exersice, the OEE cannot be calculated. 
\end{itemize}

\subsection{OEE\index{OEE}}
Which constituent does not belong in the calculation of the OEE\index{OEE}?
\begin{itemize}
	\item unplanned work interruptions
	\item slow running machines
	\item \textbf{scheduled servicing\\}
\item irregular interruptions and stoppages
\item insufficient quality of production
\end{itemize}
\subsection{EOQ\index{EOQ}}
You have product A and B in your warehouse. At the moment, you reorder 4000 units of A and 2000 of B. Calculate the EOQ with the following information and decide which statement is correct.
\begin{itemize}
\item Both products have a yearly demand of 40'000 pieces.
\item Product A has inventory holding costs of 0.5 EUR/year and unit.
\item Product B has inventory holding costs of 3 $\frac{EUR}{year}$/ and unit.
\item The ordering costs per order are 400 EUR for product A, and 600 EUR for product B. 
\item Product A and B are delivered in boxes of 500 units/box
\end{itemize}

\begin{itemize}
\item \textbf{The reordering quantities of product A and B should be doubled}\\
$EOQ A = \sqrt{\frac{2*40'000*400}{0.5}} = 8000$\\
	$EOQ B = \sqrt{\frac{2*40'000*600}{3}}= 4000$
	\item The reordering quantity of product B should be halved
	\item The reordering quantity of product A should be doubled. 
	\item The reordering quantities of product A and B should be interchanged
	\item The reordering quantities of product A and B are optimal
\end{itemize}
\subsection{EOQ\index{EOQ}}
The operator of a kiosk regularly orders lottery tickets from the countries lottery organization. His customers buy on a regular basis. He sells only one type of lottery tickets. Last year he has sold 5200 lottery tickets. One year ago he bought the kiosk including the sales premises and the operating inventory from his predecessor. He has financed the purchase with a loan. The loan is repayable at any time and carries an interest rate of 3%.

The lottery organization invoices the operator with an administrative fee of CHF 42.50 for each new order for lottery tickets. This fee covers all expenses for order administration and does not depend on the order size. The effective purchase price per lottery ticket is CHF 8.00. The sales price per lottery ticket is CHF 20.00. Any lottery winnings are paid by the countries lottery organization directly to the winners.

\textbf{How many orders per year should the operator of the kiosk place?}
\begin{itemize}
\item 2 orders per year
\item \textbf{4 orders per year}\\
$EOQ= \sqrt{\frac{2\cdot5200\cdot42.50}{0.03\cdot8}} = 1357.08 $\\
$\frac{5200}{1357.08}=3.83$ --> 4 Mal bestellen
\item 1 order every two years
\item 1 order per year
\item 8 orders per year
\end{itemize}
\subsection{Lean Management\index{Lean!Management}}
The Tim\&Friends management team has decided to introduce Lean Management (''Lean Supply Chain Management'') in addition to SCM. However the management team members are confused about all these terms and methods. Which of the following proposals is not recommendable?
\begin{itemize}
\item Optimize the change between fish and vegetables at the fryers is optimized with SMED (single minute exchange of die).
\item Preparation of a value stream map to visualize the entire flow of goods from the suppliers up to the end customers (guests in the restaurant).
\item The suppliers deliveries are optimized by applying Heijunka in order to reduce inventories.
\item The movements of the employees in the in a service process for an aircraft (e.g. at SR Technics) are visualized in a spaghetti diagram to determine waste (Muda).
\item \textbf{You make use of the ,,Enterprise Resource Planning" concept and you eliminate several unnecessary process.
steps before you automate cooking processes.}
\end{itemize}
\subsection{Lean Synchronisation\index{Lean!Synchronisation}}
Which action would you recommend a company for its improvement program with the working title "lean synchronization"?

\begin{itemize}
\item Optimization of the procurement processes for property, plant, and equipments. 
\item Changing inventory management to ensure readiness for delivery. 
\item \textbf{Demand-oriented deliveries between manufacturing steps.} 
\item Gapless logging of downtime. 
\item Improvement of data synchronization of the ERP system
\end{itemize}
\subsection{Heijunka\index{Heijunka}}
M\"uller Inc. produces tissue toilet paper. The CEO decides to implement a strategic program to increase profitability based on the lean management philosophy. He implements a concept according to the Heijunka principle (continuous flow of goods in small batches) to deliver the products from the production site to school buildings and community centers. 

Which implication do you expect?
\begin{itemize}
\item Planning in the production department becomes more challenging. 
\item The production department is able to adapt to changing customer needs less flexibly. 
\item You would not expect any of the mentioned implications. 
\item \textbf{The production department is able to adopt more flexibly to volatile demands.}
\item Planning in the procurement department becomes more challenging. 
\end{itemize}
\subsection{Six Sigma\index{Six Sigma}}
You are responsible for the production of spicy Wasabi-nuts. Recently you receive customer complaints claiming that the nuts are too spicy. Your manager does not want to take a risk and he requires you analyze the situation. A quality control identifies the following results and values: Cp = 1.44 and Cpk= 1.35. What do you report back to your manager?
\begin{itemize}
	\item You tell him that the situation is not normal and that such quality variation is unusual.
	\item The case is obvious The AQL (Acceptable Quality Level) method implies that we do not have a quality problem.
	\item Obviously a Cp-value of 1.33 is exceeded and the quality is therefore not acceptable. The products must be removed from the market.
	\item \textbf{The Cp and the Cpk value are within the typical quality standard for food items of this type (3-Sigma). The process is capable and controlled. There is no cause for concern.}
	\item The Cp-value is bigger than the Cpk-value. This means that the quality of the nuts is strongly fluctuating. Management must immediately take corrective action.
\end{itemize}
\subsection{Six Sigma\index{Six Sigma}}
 Why is it critial to reduce the variation for increasing process capability and process control?
 \begin{itemize}
 \item You have to measure more frequently.  
 \item Process control is more important than process capability. 
 \item You can adjust the measured values manually. 
 \item Process capability is a prerequisite for process improvements in general. 
 \item \textbf{Because your customers demand lower variations. }
 \end{itemize}